\section{Conclusion and Future Work}
\label{conclusion}

In this paper, we discussed why the convention approach of using a NE strategy in a Markov game might not be an optimal choice for the system defender, due to a number of assumptions about the attacker that may not hold in practice, especially in a system as complex as a smart grid. We proposed a new type of adaptive strategy called the AMS, and performed a preliminary evaluation of the technique on one class of security attacks on smart grid systems---injecting false voltage information.

Further investigation is needed to test the feasibility of our approach in practical settings. One potential limitation of the AMS, as currently designed, is the number of rounds required to converge to an accurate estimation of the attacker and obtain the best response strategy. In a smart grid, the number of packets transmitted to IEDs per second is typically in hundreds, and so in the data injection attack, it is conceivable that the AMS may converge to an optimal strategy in matter of minutes. However, other types of attacks that are less frequent (e.g., where an attacker's action involves physical hampering), a dynamic technique such as the AMS might not be suitable. We plan to investigate possible ways to reduce the duration of the convergence (e.g., using an approximation). Furthermore, we plan to study the effectiveness of the AMS on other types of smart grid attacks, and explore techniques for scaling the computation of the AMS to larger distribution systems.
