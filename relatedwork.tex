\section{Related Work}
\label{relatedwork}
Game theory has been widely used as a mathematical tool to model and analyse the security issues in critical infrastructures such as smart-grid systems \cite{saad2012game,brown2006defending,salmeron2004analysis,pinar2010optimization}. The interaction between the defender and attacker is usually modeled as a single-shot Stackelberg game, in which the defender and attacker are considered as the leader and follower and make sequential moves. The goal thus is to identify the optimal defending strategy (e.g., the most critical set of components to protect) for the defender to minimize the potential loss of the system. However, in practice the defender and attacker may interact with each other repeatedly and the system evolves dynamically depending on their actions.

Markov games \cite{minmaxQ} later are adopted to model the repeated strategic interactions between the defender and attacker in smart-grid systems. In \cite{ma2013markov}, one specific physical attack on the transmission lines of the smart-grid system is considered and the interaction between the defender and attacker is modeled as a Markov game, in which the system states (the status of the transmission lines) evolves based on their joint actions. A NE solution is adopted as the defending strategy for the system, which specifies which transmission lines to protect. In \cite{law2012security}, one specific cyber attack (false data injection attack) is studied and the repeated attacker-defender cyber-interaction is modeled as a Markov game. Similar to \cite{ma2013markov}, the NE solution is adopted as the defending strategy which determines which action to choose to perform intrusion detection. However, adopting the NE solution as the defending strategy is rational only when the attacker is also choosing the corresponding NE strategy to launch the attack, which may not hold in practice.

