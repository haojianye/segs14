One active area of research in smart grid security focuses on applying
game-theoretic frameworks to analyze interactions between a system and
an attacker and formulate effective defense strategies. In previous work,
a Nash equilibrium (NE) solution is chosen as the optimal defense
strategy \cite{law2012security,ma2013markov}, which implies that the
attacker has complete knowledge of the system and would also employ
the corresponding NE strategy. In practice, however, the attacker may
have limited knowledge and resources, and thus employ an attack which
is less than optimal, allowing the defender to devise more efficient
strategies.

We propose a novel approach called an \underline{a}daptive
\underline{M}arkov \underline{s}trategy (AMS) for defending a system
against attackers with unknown, dynamic behaviors. The algorithm for
computing an AMS is theoretically guaranteed to converge to a best
response strategy against any stationary attacker, and also converge
to a Nash equilibrium if the attacker is sufficiently intelligent to
employ the AMS to launch the attack. To evaluate the effectiveness of
an AMS in smart grid systems, we study a class of data integrity
attacks that involve injecting false voltage information into a
substation, with the goal of causing load shedding (and potentially a
blackout). Our preliminary results show that the amount of load
shedding costs can be significantly reduced by employing an AMS over
a NE strategy.
